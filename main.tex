% \iffalse
\let\negmedspace\undefined
\let\negthickspace\undefined
\documentclass[journal,12pt,twocolumn]{IEEEtran}
\usepackage{cite}
\usepackage{amsmath,amssymb,amsfonts,amsthm}
\usepackage{algorithmic}
\usepackage{graphicx}
\usepackage{textcomp}
\usepackage{xcolor}
\usepackage{txfonts}
\usepackage{listings}
\usepackage{enumitem}
\usepackage{mathtools}
\usepackage{gensymb}
\usepackage{comment}
\usepackage[breaklinks=true]{hyperref}
\usepackage{tkz-euclide} 
\usepackage{listings}
\usepackage{gvv}                                        
\def\inputGnumericTable{}                                 
\usepackage[latin1]{inputenc}                                
\usepackage{color}                                            
\usepackage{array}                                            
\usepackage{longtable}                                       
\usepackage{calc}                                             
\usepackage{multirow}                                         
\usepackage{hhline}                                           
\usepackage{ifthen}                                           
\usepackage{lscape}

\newtheorem{theorem}{Theorem}[section]
\newtheorem{problem}{Problem}
\newtheorem{proposition}{Proposition}[section]
\newtheorem{lemma}{Lemma}[section]
\newtheorem{corollary}[theorem]{Corollary}
\newtheorem{example}{Example}[section]
\newtheorem{definition}[problem]{Definition}
\newcommand{\BEQA}{\begin{eqnarray}}
\newcommand{\EEQA}{\end{eqnarray}}
\newcommand{\define}{\stackrel{\triangle}{=}}
\theoremstyle{remark}
\newtheorem{rem}{Remark}
\begin{document}
\parindent 0px

\bibliographystyle{IEEEtran}
\vspace{3cm}

\title{ASSIGNMENT-1}
\author{AI24BTECH11026 - Pendem nitesh sri satya$^{*}$% <-this % stops a space
}
\maketitle
\newpage
\bigskip

\renewcommand{\thefigure}{\theenumi}
\renewcommand{\thetable}{\theenumi}
\begin{enumerate}

\setcounter{enumi}{15}

 \item Let $a$, $b$, $c$ be real numbers, a $\neq$ 0. If $\alpha$ is a root of $a^2x^2$+$bx$+$c$=0. $\beta$ is the root of $a^2x^2$-$bx$-$c$=0 and 0$<$$\alpha$$<$$\beta$, then the equation $a^2x^2$+$2bx$+$2c$=0 has a root $\gamma$ that always satisfies 

(a) $\gamma$= $\frac{\alpha+\beta}{2}$  \hfill (1989 - 2$ Marks$)

(b) $\gamma$=  $\alpha$+$\frac{\beta}{2}$

(c) $\gamma$= $\alpha$

(d) $\alpha$$<$$\gamma$$<$$\beta$

 \item The number of solutions of the equation sin$(e)^x$ = $5^x$+$5^{-x}$ is 

(a) 0 \hfill (1990 - 2 $Marks$)

(b) 1

(c) 2

(d) Infinitely many

 \item Let $\alpha$,$\beta$ be the roots of the equation 

($x-a$)($x-b$)=$c$, $c \neq 0$ Then the roots of the equation ($x-a$)($x-b$)+$c$=0 are 

(a) $a,c$ \hfill (1992 - 2 $Marks$)

(b) $b,c$

(c) $a,b$

(d) $a+c,b+c$

 \item The number of point of intersection of two curves $y$=2sin$x$ and $y$=$5x^2+2x+3$ is 

(a) 0 \hfill (1994)

(b) 1

(c) 2

(d) $\infty$

 \item If $p,q,r$ are +ve and are in $A.P$.,the roots of quadratic equation $px^2+qx+r$ = 0 are all real for

(a) $\left|\frac{r}{p}-7\right|$ $\geq$ 4$\sqrt{3}$ \hfill (1994)

(b) $\left|\frac{p}{r}-7\right|$ $\geq$ 4$\sqrt{3}$

(c) all $p$ and $r$

(d) no $p$ and $r$

 \item Let $p,q \in$  ${1, 2, 3, 4}$. The number of equations of the form $px^2+qx+1$ = 0 having real roots is

(a) 15 \hfill (1994)

(b) 9

(c) 7

(d) 8

 \item If the roots of the equation 

$x^2-2ax+a^2+a-3$ = 0 are real and less than 3, then \hfill (1999 - 2 $Marks$)

(a) $a<2$

(b) 2 $\leq$ $a$ $\leq$ 3

(c) 3 $<$ $a$ $\leq$ 4p

(d) $a$ $>$ 4

 \item If $\alpha$ and $\beta$ ($\alpha < \beta$) are the roots of the equation $x^2+bx+c$ = 0, where $c < 0 < b$, then \hfill (2000$S$)

(a) $0 < \alpha < \beta$ 

(b) $\alpha < 0 < \beta < |\alpha|$

(c) $\alpha < \beta <0$

(d) $\alpha < 0 < |\alpha| < \beta$

 \item If $a, b, c, d$ are positive real numbers such that $a+b+c+d$ = 2, then $M$ = ($a+b$)($c+d$) satifies the relation \hfill (2000$S$)

(a) both roots in ($a,b$)

(b) both roots in (-$\infty$, $a$)

(c) both roots in ($b$, +$\infty$)

(d) one root in (-$\infty$, $a$) and the other in ($b$, +$\infty$)

 \item If $b > a$, then the equation ($x-a$)($x-b$)-1 = 0 has \hfill (2000$S$)

(a) both roots in ($a,b$)

(b) both roots in (-$\infty$,$a$)

(c) both roots in ($b,+\infty$

(d) one root in ($-\infty,a$) and the other in ($b,+\infty$)

 \item  For the equation $3x^2+px+3$ = 0,$p>$ 0, if one of the root is square of the other, then $p$ is equal to \hfill (2000$S$)

(a) $\frac{1}{3}$

(b) 1

(c) 3

(d) $\frac{2}{3}$

 \item  If $a_1,a_2.....,a_n$ are positive real numbers whose product is a fixed number c, then the minimum value of $a_1+a_2+.......+a_{n-1}+2a_n$ is \hfill (2002$S$)

(a) $n(2c)^{\frac{1}{n}}$

(b) $(n+1)c^\frac{1}{n}$
 
(c) $2nc^\frac{1}{n}$

(d) $(n+1)(2n)^\frac{1}{n}$

 \item The set of all real numbers $x$ for which 

$x^2-|x+2|+x$ $>$ 0, is \hfill (2002$S$)

(a) (-$\infty$,-2) $\cup$ (2,$\infty$)

(b) ($-\infty,-\sqrt{2}$) $\cup$ ($\sqrt{2},\infty$)

(c) ($-\infty,-1$) $\cup$ ($1,\infty$)

(d) ($\sqrt{2},\infty$)

 \item  If $\alpha$ $\in$ (0,$\frac{\pi}{2}$) then $\sqrt{x^2+x}$+$\frac{tan^2\alpha}{\sqrt{x^2+x}}$ is always greater than or equal to \hfill (2003$S$)

(a) 2$tan\alpha$

(b) 1

(c) 2

(d) $sec^2\alpha$

 \item For all $'x'$ ,$x^2+2ax+10-3a >0$, then the interval in which $'a'$ lies is \hfill (2004$S$)

(a) $a < -5$

(b) $-5 < a < 2$

(c) $a > 5$

(d) $2 < a < 5$









\end{enumerate}
\end{document}

